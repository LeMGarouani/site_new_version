\documentclass[9pt]{beamer}
%\documentclass{bredelebeamer}
\usepackage[utf8]{inputenc}


\usetheme[progressbar=frametitle]{metropolis}
\setbeamercolor{background canvas}{bg=white}
\usepackage{appendixnumberbeamer}


\defbeamertemplate*{headline}{smoothbars theme}
{%
\begin{beamercolorbox}[ht=2.125ex,dp=3.150ex]{section in head/foot}
\insertnavigation{\paperwidth}
\end{beamercolorbox}%

% Commenter les 4 lignes suivantes si vous ne voulez pas la barre des sous-sections.
%\begin{beamercolorbox}[ht=2.125ex,dp=1.125ex,%
%leftskip=.3cm,rightskip=.3cm plus1fil]{subsection in head/foot}
%\usebeamerfont{subsection in head/foot}\insertsubsectionhead
%\end{beamercolorbox}%
}


%\usepackage{textpos}
%\usepackage{pdfpc-commands}
\usepackage{multicol}
\usepackage{media9}
\usepackage{xmpmulti}
\usepackage{animate}
\usepackage{fancyhdr}
\usepackage{graphicx}
\usepackage{caption}
\usepackage{subcaption}
\usepackage{comment}
\usepackage{colortbl}
\captionsetup[subfigure]{font=scriptsize,labelfont=scriptsize}
%\usepackage{subfig}
\usepackage{xcolor}
\usepackage[export]{adjustbox}

\usepackage{tabularx} % in the preamble
\usepackage{lipsum}% http:#ctan.org/pkg/lipsum
\usepackage{csquotes}

\usepackage{booktabs}
\usepackage[scale=2]{ccicons}

\usepackage{pgfplots}
\usepgfplotslibrary{dateplot}
\usepackage[citestyle=authoryear]{biblatex}
\usepackage[export]{adjustbox}

\usepackage{tikz}

\usepackage{xspace}
\usepackage{array}
\usepackage{verbatim} 
\newcommand{\themename}{\textbf{\textsc{metropolis}}\xspace}

\DeclareUnicodeCharacter{0301}{\'{e}}

\usepackage{tikz}
\usetikzlibrary{shapes.arrows, arrows, chains, positioning, decorations.markings}

\usepackage[absolute, overlay]{textpos}

\usepackage[french]{babel}
\usepackage{listings}
\usepackage{listingsutf8}
%\usepackage{multimedia}

\newcommand{\includemovie}[3]{%
    \includemedia[%
    width=#1,height=#2,%
    activate=pageopen,
    passcontext,  %show VPlayer's right-click menu
    addresource=#3,%
    flashvars={%
        source=#3 % same path as in addresource!
        &autoPlay=true % default: false; if =true, automatically starts playback after activation (see option ‘activation)’
        &loop=true % if loop=true, media is played in a loop
        &controlBarAutoHideTimeout=0 %  time span before auto-hide
    }%
    ]{\fbox{Loading video...}}{VPlayer.swf}%
}% end of the new command


\definecolor{pblue}{rgb}{0.13,0.13,1}
\definecolor{pgreen}{rgb}{0,0.5,0}
\definecolor{pred}{rgb}{0.9,0,0}
\definecolor{pgrey}{rgb}{0.46,0.45,0.48}

\definecolor{violetschema}{RGB}{143, 143, 255}
\definecolor{ulcoblue}{RGB}{0, 107, 179}
\definecolor{customgreen}{RGB}{15, 187, 15}

\lstset{language=C++,extendedchars=true,inputencoding=latin1,
    basicstyle=\ttfamily\small, commentstyle=\ttfamily\color{green!150!black},
      showstringspaces=false,basicstyle=\ttfamily\small}

\lstdefinestyle{custombash}{
	inputencoding=utf8,
	mathescape,
	basicstyle=\footnotesize\ttfamily,
	language=bash,
	breaklines=true,
	frame             = lines,
	commentstyle=\ttfamily\color{pgreen},
%	framexleftmargin = 8mm,
	framerule         = 1pt,
	rulecolor         = \color{gray!50!black},
	xleftmargin=\parindent,
	extendedchars=true, 
	breaklines=true,
	breakatwhitespace=true,
	backgroundcolor = \color{gray!6},
	literate=
	{é}{{\'e}}{1}%
	{è}{{\`e}}{1}%
	{à}{{\`a}}{1}%
	{â}{{\^a}}{1}%%%
	{ç}{{\c{c}}}{1}%
	{œ}{{\oe}}{1}%
	{ù}{{\`u}}{1}%
	{É}{{\'E}}{1}%
	{È}{{\`E}}{1}%
	{À}{{\`A}}{1}%
	{Ç}{{\c{C}}}{1}%
	{Œ}{{\OE}}{1}%
	{Ê}{{\^E}}{1}%
	{ê}{{\^e}}{1}%
	{î}{{\^i}}{1}%
	{ï}{{\"i}}{1}%%%
	{ô}{{\^o}}{1}%
	{û}{{\^u}}{1}%
	%{=}{$\leftarrow$}{1}%
	%{==}{$={}$}{1}
}

\lstdefinestyle{customc}{
	inputencoding=utf8,
	belowcaptionskip=1\baselineskip,
	breaklines=true,
	frame=L,
	xleftmargin=\parindent,
	language=C,
	showstringspaces=false,
	basicstyle=\footnotesize\ttfamily,
	keywordstyle=\bfseries\color{green!40!black},
	commentstyle=\itshape\color{purple!40!black},
	identifierstyle=\color{blue},
	stringstyle=\color{orange},
}

\title{Partie 1 : Introduction à \texttt{git}}
\subtitle{Comment collaborer et versionner correctement son projet}
\date{\today}

\author{Jérôme Buisine (PhD Student) \\
%\textbf{Supervisors:} Christophe Renaud and Samuel Delepoulle\\
\textbf{Team:} IMAP (Images et Apprentissage)}

	%, Christophe Renaud\inst{1}, Samuel Delepoulle\inst{1} and Philippe Preux\inst{2}}

% La commande \inst{...} Permet d'afficher l' affiliation de l'intervenant.
% Si il y a plusieurs intervenants: Marcel Dupont\inst{1}, Roger Durand\inst{2}
% Il suffit alors d'ajouter un autre institut sur le modèle ci-dessous.

% \institute[Université du Littoral Côte d'Opale]
% {
% 	\inst{*}
% 	ANR support : project ANR-17-CE38-0009 \\
% 	\inst{}%
% 	Univ. Littoral Côte d’Opale, LISIC, F-62100 Calais, France \\
% 	% \includegraphics[width=0.4\textwidth]{images/0.Logos/logo_ulco_lisic.png}
% }


% \titlegraphic{
% 	\hfill\includegraphics[height=1.2cm]{images/0.Logos/all_logo_with_anr.png}
% }

% \setbeamerfont{section in toc}{size=\small}
% \setbeamerfont{subsection in toc}{size=\footnotesize}
% \addtobeamertemplate{frametitle}{}{%
% 	\begin{textblock*}{100mm}(.1\textwidth,1.07\textheight)
% 		\includegraphics[width=0.18\textwidth]{images/0.Logos/logo_lisic_anr_light.png}
% \end{textblock*}}

%\setbeamertemplate{section in toc}[sections numbered]
%\setbeamertemplate{subsection in toc}[subsections numbered]

% \AtBeginSubsection[]

%   \begin{frame}
%       \frametitle{Outline}
%        %\tableofcontents[currentsection,currentsubsection]
%        \tableofcontents[currentsection]
%    \end{frame}
%}

%\addtobeamertemplate{title page}{}{%
%	\begin{textblock*}{100mm}(.5\textwidth,0.5\textheight)
%		\includegraphics[width=0.2\textwidth]{images/0.Logos/logo_ulco_lisic.png}
%\end{textblock*}}


%\logo{\includegraphics[width=0.3\textwidth]{images/0.Logos/logo_lisic_anr_light.png}}

%\bibliography{CSI-SecondYear-Slides.bib}

\begin{document}
	
	\maketitle
	
	\begin{frame}{Sommaire}
		\setbeamertemplate{section in toc}[sections numbered]
		\tableofcontents[hideallsubsections]
	\end{frame}

\section{Gestion de version}
	    
	%%\subsection{MC noise in synthesis images}
	\begin{frame}[fragile]{Un problème commun ?}
		
        Trouver un nommage pour versionner un document n'est pas évident :

        \vspace{5mm}
        \begin{lstlisting}[style=custombash]
         Mémoire_v1.doc
         Mémoire_v2.doc
         Mémoire_v3.doc
         Mémoire_Final.doc
         Mémoire_Final_v2.doc
         Mémoire_Final_Final.doc
         ...
         Mémoire_Final_pour_de_vrai.doc
        \end{lstlisting}
        
	\end{frame}

    \begin{frame}[fragile]{Parlons convention}
        
        Une \textbf{convention} dans un premier temps ?

        \vspace{5mm}
        \begin{lstlisting}[style=custombash]
         Mémoire_v0.0.1.doc
         Mémoire_v0.0.2.doc
         Mémoire_v0.1.0.doc
         Mémoire_v0.2.0.doc
         Mémoire_v1.0.0.doc
         Mémoire_v1.0.1.doc
         ...
         Mémoire_v2.0.0.doc
        \end{lstlisting}

    \end{frame}

    \begin{frame}{Problème de collaboration}

        \begin{overprint}
            
        \onslide<1->{
        Comment gérer le maintien du document avec plusieurs personnes ?
        }

        \onslide<2->{
        \begin{itemize}
            \item Qui travaille actuellement et sur quelle version ?
            \item Quelle est réellement la dernière version ?
        \end{itemize}
        }
    
        \onslide<3->{
        \ifthenelse{\boolean{beamer@anotherslide}}{
            \phantom{\rule{0.5\textwidth}{0.45\textheight}}
        }{
            \begin{figure}[ht]
                \centering
                \includemovie{0.36\textwidth}{0.25\textheight}{resources/slides/mind_fuck.mp4}%
                \caption{\footnotesize{Visage de la personne désignée volontaire pour la fusion du document}}
            \end{figure}  
        }
        }
        
        \onslide<4->{
        \metroset{block=fill}
        \begin{alertblock}{Édition de document en ligne}
            Outils \textbf{collaboratifs} permettant d'éditer en ligne et en parallèle le même fichier aisément
        \end{alertblock}
        }
        
        \end{overprint}
    \end{frame}

\section{Vers des outils collaboratifs}


    \begin{frame}{Outils collaboratifs}
        
        \textbf{Pour les développeurs :}
        \vspace{5mm}

        \begin{figure}
            \centering
            \begin{subfigure}[b]{0.4\textwidth}
                \centering
                \includegraphics[width=0.4\textwidth]{resources/slides/svn_logo.jpg}
                \caption{Fondation Apache en 2004}
            \end{subfigure}
            ~
            \begin{subfigure}[b]{0.4\textwidth}
                \centering
                \includegraphics[width=0.5\textwidth]{resources/slides/git_logo.png}
                \caption{Linus Thorvald en 2008}
            \end{subfigure}
        \end{figure}
    \end{frame}

    \begin{frame}{L'outil collaboratif \texttt{git}}

        \begin{overprint}
            
            \textbf{Particularités :}
            \onslide<1->{
                \begin{itemize}
                    \item Système de contrôle de version décentralisé
                    \item Chaque participant possède un clone de l’ensemble du référentiel en local
                    \item Ajouter, contribuer et suivre les changements dans le code source
                \end{itemize}
            }
            
            \onslide<2->{
                \begin{figure}[ht]
                    \centering
                    \includegraphics[width=0.6\textwidth]{resources/git_classical.png}
                    
                    \caption{Échanges bilatéraux de développeurs (source : \href{https://git-scm.com/about/distributed}{git-scm}).}
                    \label{fig:git_classical}
                \end{figure}
            }
        \end{overprint}
        
    \end{frame}

\section{Utilisation basique de \texttt{git}}

    \begin{frame}[fragile]{Interface de commandes}

        \textbf{Commandes \texttt{git} de base :}

        \begin{lstlisting}[style=custombash]
         # Initialisation du projet
         git init

         # Ajout de fichier dans l'index
         git add file.txt

         # Vérification de l'index
         git status

         # Suppression d'un fichier dans l'index
         git reset file.txt
        \end{lstlisting}

        
        \begin{figure}[ht]
            \centering
            \includemovie{0.8\textwidth}{0.235\textheight}{resources/slides/git_basic.mp4}%
        \end{figure}  
        
    \end{frame}

    \begin{frame}[fragile]{Ajout de modifications}

        
        \begin{figure}[ht]
            \centering
            \begin{subfigure}{.395\textwidth}
                \includegraphics[width=\textwidth]{resources/git_add.png}
                %\caption{Ajout de modifications à l’index en utilisant le contenu actuel trouvé puis validation des modifications à l'aide d'un \texttt{commit}.}
            \end{subfigure}
            ~
            \begin{subfigure}{.415\textwidth}
                \includegraphics[width=\textwidth]{resources/git_commit.png}
                %\caption{Ajout de toutes les modifications courantes à l’index avec validation directe des modifications à l'aide d'un \texttt{commit}.}
            \end{subfigure}
            \caption{Ajout et validation de contenu de fichiers (source : \href{https://git-scm.com/about/staging-aread}{git-scm})}
            \label{fig:git_add_command}
        \end{figure}
        
    \end{frame}

    \begin{frame}[fragile]{Gestion des modifications}

        \textbf{Ajout de contenu :}

        \begin{lstlisting}[style=custombash]
         # Validation de l'index (fichiers ajoutés)
         git commit -m "some updates"

         # Ajout des modifications courantes et validation
         git commit -am "some updates"

         # Aperçu des commits
         git log

         # Affichage condensé des commits
         git log --oneline
        \end{lstlisting}

        \begin{figure}[ht]
            \centering
            \includemovie{0.8\textwidth}{0.235\textheight}{resources/slides/git_commit.mp4}%
        \end{figure}  
        
    \end{frame}

    \begin{frame}[fragile]{Suppression de modifications}

        \textbf{Suppressions locales :}

        \begin{lstlisting}[style=custombash]
        # Supprimer le dernier commit local
        git reset --soft HEAD^
        
        # Supprimer les 2 derniers commits locaux
        git reset --soft HEAD~2
        \end{lstlisting}

        \begin{figure}[ht]
            \centering
            \includemovie{0.8\textwidth}{0.235\textheight}{resources/slides/git_reset.mp4}%
        \end{figure}  
        
    \end{frame}

\section{Notions importantes}

    \begin{frame}[fragile]{Version de projet}

        \textbf{Numéros de version sous la forme \texttt{X.Y.Z}}

        \vspace{2mm}
        \begin{itemize}
            \item[$\bullet$]  X – Majeur : suppression d’une fonctionnalité obsolète, modification d’interfaces, renommages...
            \item[$\bullet$]  Y – Mineur : introduction de nouvelles fonctionnalités, fonctionnalité marquée comme obsolète...
            \item[$\bullet$]  Z – Correctif : modification/correction d’un comportement interne, failles de sécurité...
        \end{itemize}
        
    \end{frame}

    \begin{frame}[fragile]{Ajout de version}
        
        \textbf{Réaliser une livraison avec version :}
        \begin{lstlisting}[style=custombash]
        # Ajout d'un tag de version à l'état actuel du projet
        git tag -a v1.0.0 -m "Releasing version v1.0.0"

        # Liste l'ensemble des versions du projet
        git tag -l

        # Affiche les informations détaillées d'un tag
        git show v1.0.0
        \end{lstlisting}

    \end{frame}

    \begin{frame}[fragile]{Gestion des branches}

        \textbf{Vers l'utilisation de branches ??}
        
        \begin{picture}(0,0)
            \put(240.5,-8.45){
                \includemovie{0.25\textwidth}{0.15\textheight}{resources/slides/gif_wow.mp4}
        }
        \end{picture}

        \pause
        \begin{figure}[ht]
            \centering
            \includegraphics[width=0.75\textwidth]{resources/branch_git.png}
            
            \caption{Exemple de gestion des branches sous \texttt{git} (source : \href{https://www.qovery.com/blog/3-ways-of-cloning-an-application-and-a-database-per-git-branch}{qovery.com}).}
            \label{fig:git_branch}
        \end{figure}
        
    \end{frame}

    \begin{frame}[fragile]{Gestion des branches}
        
        \textbf{Commandes pour la gestion des branches :}
        \begin{lstlisting}[style=custombash]
            # Liste toutes les branches disponibles
            git branch
            
            # Crée une nouvelle branche
            git branch ma-branche
           
            # Change de branche de développement
            git checkout ma-branche
               
            # Création et changement de branche courante
            git checkout -b ma-branche
            
            # Supprime une branche
            git branch -d ma-branche
            
            # Depuis la branche `develop`, récupération des modifications de `ma-branche`
            # Plus évident à manipuler que rebase
            git merge ma-branche
        \end{lstlisting}

    \end{frame}

    \begin{frame}[fragile]{Serveur d'hébergement}

        \begin{figure}
            \includegraphics[width=0.5\textwidth]{resources/slides/git_server.jpg}
        \end{figure}

        \pause
        \textbf{Interaction avec le serveur d'hébergement :}
        \begin{lstlisting}[style=custombash]
        # Énumère les serveurs d'hébergements référencés
        git remote -v

        # Récupère et applique les modifications du serveur d'hébergement
        git pull <remote-name> ma-branche

        # Publie les modifications apportés par une branche sur un serveur
        # !! toujours réaliser un pull avant de push !!
        git push <remote-name> ma-branche
        \end{lstlisting}
                
    \end{frame}

    \begin{frame}[fragile]{Autres commandes}

        \textbf{Commandes à ne pas confondre :}

        \begin{itemize}
            \item \textbf{git revert} : crée un nouveau commit qui annule les changements d'un commit précédent ;
            \item \textbf{git checkout} : extrait le contenu du référentiel et le place dans votre arbre de travail (elle permet aussi le déplacement vers une autre branche) ;
            \item \textbf{git reset} : il modifie l'index (la \textquote{zone de transit}). Ou elle change le commit sur lequel une tête de branche pointe actuellement. Cette commande peut modifier l'historique existant. 
        \end{itemize}

    \end{frame}

    \begin{frame}[fragile]{Autres commandes}

    \textbf{Cas d'utilisations possibles :}

        \begin{itemize}
            \item \textbf{git revert} : si un commit a été fait quelque part dans l'historique du projet, et que vous décidez plus tard que ce commit est mauvais. L'utilisation de \textit{git revert} annulera les changements introduits par le mauvais commit, en enregistrant le \textquote{undo} dans l'historique ;
            \item \textbf{git checkout} : restaure une révision historique d'un fichier donné (\texttt{git checkout <file-path> <commit-hash>}) ;
            \item \textbf{git reset} : si vous avez fait un commit, mais que vous ne l'avez pas partagé avec quelqu'un d'autre et que vous décidez que vous n'en voulez pas, alors vous pouvez utiliser git reset pour réécrire l'historique de sorte qu'il semble que vous n'ayez jamais fait ce commit. 
        \end{itemize}

    \end{frame}
	
\end{document}